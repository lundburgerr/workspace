%______________________________________________________
%
%   LaTeX-mall fr nybrjare
%
%   Konstruerad av Marcus Bergner, bergner@cs.umu.se
%
%   Vid funderingar titta lngst ned i denna fil,
%   eller skicka ett mail
%______________________________________________________
%

% lite instllningar
\documentclass[10pt, titlepage, oneside, a4paper]{article}
\usepackage{amsmath}
\usepackage[T1]{fontenc}
\usepackage[utf8]{inputenc}
\usepackage[english]{babel}
\usepackage{amssymb, graphicx, fancyheadings}
\usepackage{placeins}

%For subfigures
\usepackage{caption}
\usepackage{subcaption}

\addtolength{\textheight}{20mm}
\addtolength{\voffset}{-5mm}
\renewcommand{\sectionmark}[1]{\markleft{#1}}

% \Section ger mindre spillutrymme, anvnd dem om du vill
\newcommand{\Section}[1]{\section{#1}\vspace{-8pt}}
\newcommand{\Subsection}[1]{\vspace{-4pt}\subsection{#1}\vspace{-8pt}}
\newcommand{\Subsubsection}[1]{\vspace{-4pt}\subsubsection{#1}\vspace{-8pt}}
	
% appendices, \appitem och \appsubitem r fr bilagor
\newcounter{appendixpage}

\newenvironment{appendices}{
	\setcounter{appendixpage}{\arabic{page}}
	\stepcounter{appendixpage}
}{
}

\newcommand{\appitem}[2]{
	\stepcounter{section}
	\addtocontents{toc}{\protect\contentsline{section}{\numberline{\Alph{section}}#1}{\arabic{appendixpage}}}
	\addtocounter{appendixpage}{#2}
}

\newcommand{\appsubitem}[2]{
	\stepcounter{subsection}
	\addtocontents{toc}{\protect\contentsline{subsection}{\numberline{\Alph{section}.\arabic{subsection}}#1}{\arabic{appendixpage}}}
	\addtocounter{appendixpage}{#2}
}

% ndra de rader som behver ndras
\def\inst{Department of computer science}
\def\typeofdoc{Lab report}
\def\course{Computer Organization and Architecture 7.5hp}
\def\pretitle{Lab 1}
\def\title{Mips32 disassembler}
\def\name{Robin Lundberg}
\def\username{ens10rlg}
\def\email{\username{}@cs.umu.se}
\def\path{/home/rolund03/workspace/gsl_ode}
\def\graders{Stephen J. Hegner, Thomas Johansson}


% om du vill referera till katalogen dr dina filer ligger kan du 
% anvnda \fullpath som kommer att vara "~username/edu..." o.s.v.
%\def\fullpath{\raisebox{1pt}{$\scriptstyle \sim$}\username/\path}


% Hr brjar sjlva dokumentet
\begin{document}

	% skapar framsidan (om den inte duger: gr helt enkelt en egen)
	\begin{titlepage}
		\thispagestyle{empty}
		\begin{large}
			\begin{tabular}{@{}p{\textwidth}@{}}
				\textbf{UMEÅ UNIVERSITY \hfill \today} \\
				\textbf{\inst} \\
				\textbf{\typeofdoc} \\
			\end{tabular}
		\end{large}
		\vspace{10mm}
		\begin{center}
			\LARGE{\pretitle} \\
			\huge{\textbf{\course}}\\
			\vspace{10mm}
			\LARGE{\title} \\
			\vspace{15mm}
			\begin{large}
				\begin{tabular}{ll}
					\textbf{Name} & \name \\
					\textbf{E-mail} & \texttt{\email} \\
					%\textbf{Path} & \texttt{/home/rolund03/workspace/gsl\_ode} \\
				\end{tabular}
			\end{large}
			\vfill
			\large{\textbf{Supervisor}}\\
			\mbox{\large{\graders}}
		\end{center}
	\end{titlepage}


	% fixar sidfot
	\lfoot{\footnotesize{\name, \email}}
	\rfoot{\footnotesize{\today}}
	\lhead{\sc\footnotesize\title}
	\rhead{\nouppercase{\sc\footnotesize\leftmark}}
	\pagestyle{fancy}
	\renewcommand{\headrulewidth}{0.2pt}
	\renewcommand{\footrulewidth}{0.2pt}

	% skapar innehllsfrteckning.
	% Tnk p att kra latex 2ggr fr att uppdatera allt
	\pagenumbering{roman}
	%\tableofcontents
	
	% och lägger in en sidbrytning
	\newpage

	\pagenumbering{arabic}

	% i Sverige har vi normalt inget indrag vid nytt stycke
	\setlength{\parindent}{0pt}
	% men dremot lite mellanrum
	\setlength{\parskip}{10pt}

	% lägger in rubrik (finns \section, men då får man mycket spillutrymme)
	\FloatBarrier
	\Section{Problem Specification}
	In the MIPS32 architecture, all machine code instructions are represented as 32-bit instructions.
The goal of this lab is to write a \emph{disassembler} in \emph{C} that turns these machine code instructions into MIPS32 assembly language.
The program should read a file as input that contains the instructions in decimal or hexadecimal---one in each row---for the MIPS32-program. For every line in the machine code file; the disassembler should output a line that contains the following:
\begin{itemize}
\item the original instruction.
\item the type of instruction.
\item the values for relevant fields for that type of instruction in both decimal and hexadecimal.
\item and the mnemonic representation.
\end{itemize}

The disassembler should handle all instructions, except floating point instructions.
If the instruction is not valid the program should output relevant information for that line.

	\FloatBarrier
	\Section{Usermanual}
	%1. How to compile the program
To compile the disassembler, run \texttt{make dismips}.
%2. How to run the program
Run the disassembler by executing the command \texttt{dismips file}.
Where \texttt{file} is the path to the file containing the MIPS32 machine code to be disassembled.
The disassembler supports all instructions except those with op-code 16, 17 and 18 except for the commands
\texttt{bclf}, \texttt{bclt}, \texttt{mfc1} and \texttt{mtc1} which are included.

%3. How is the output formatted
The disassembler outputs to \emph{standard out} the following for each line of machine code:
\begin{enumerate}
\item the original instruction.
\item the type of instruction.
\item the values for relevant fields for that type of instruction in decimal.
\item the values for relevant fields for that type of instruction in hexadecimal.
\item and the mnemonic representation.
\end{enumerate}
in the format \texttt{1; 2; 3; 4; 5;} where the numbers corresponds to the information in the list above.
The disassembler will output error messages that corresponds to whatever type of error is in the machine code e.g. if you try to use a floating point instruction it will tell you it is not allowed or if you use an invalid op- or function-code it will also tell you that---and then continue with the next line of machine code.		
	
	\FloatBarrier

%	% här brjar alla bilagor. Denna måste finnas med även om bara
%	% bilagor anges i \begin{appendices} ... \end{appendices}
%	\appendix
%
%	\Section{Bilaga 1}
%	\ldots{}ligger direkt i dokumentet
%
%	% bilagor, t.ex. källkod. En tom extrasida kommer att skrivas ut för
%	% att få alla sidnummer att stämma
%	\begin{appendices}
%		\appitem{Källkod}{0}
%		\appsubitem{\texttt{mish.c}}{2}
%		\appsubitem{\texttt{mish.h}}{1}
%		\appitem{En bilaga på 3 sidor}{3}
%	\end{appendices}

\end{document}


% Lite information om hur man arbetar med LaTeX
%-----------------------------------------------
%
% LaTeX-koden kan skrivas med en godtycklig editor.
% Fr att "kompilera" dokumentet anvnds kommandot latex:
%    bergner@peppar:~/edu/sysprog/lab1> latex rapportmall.tex
% Resultatet blir ett antal filer, bl.a. en som heter rapportmall.dvi.
% Denna fil kan anvndas fr att titta hur dokumentet egentligen ser
% ut med hjlp av programmet xdvi:
%    bergner@peppar:~/edu/sysprog/lab1> xdvi rapportmall.dvi &
% Du fr d upp ett fnster som visar ditt dokument. Detta fnster
% kommer automatiskt att uppdateras d du ndrar och kompilerar om din
% LaTeX-kod. 
% Nr du anser att din rapport r frdig att skrivas ut anvnder man
% lmpligtvis kommandona dvips och lpr:
%    bergner@peppar:~/edu/sysprog/lab1> dvips -P ma436ps rapportmall.dvi
% Om man vill ha kvar PostScript-filen som dvips genererar kan man gra:
%    bergner@peppar:~/edu/sysprog/lab1> dvips -o rapport.ps rapportmall.dvi
%    bergner@peppar:~/edu/sysprog/lab1> lpr -P ma436ps rapport.ps
% OBS!!! Fr att innehllsfrteckningen och eventuella referenser till
% tabeller och figurer garanterat ska stmma mste man kra latex 2ggr
% p sitt dokument efter att man har ndrat ngot.
%
%
% Lite information om saker man kan tnkas behva i sitt arbete med LaTeX
%-------------------------------------------------------------------------
%
% FORMATTERA TEXT
%
% Fr att formattera text p lite olika stt kan man anvnda fljande LaTeX-
% kommandon:
%    \textbf{denna text kommer att vara i fetstil}
%    \emph{denna text r viktig (kursiv stil)}
%    \texttt{i denna text blir alla tecken lika breda, som med en skrivmaskin}
%    \textsf{denna text visas med ett typsnitt utan serifer}
%
%
% MATEMATISKA FORMLER
%
% Fr att typstta matematiska formler kan man anvnda:
%    $f(x) = x^2 - 3$, vilket lgger in formeln i texten, eller
%    \begin{displaymath}
%        g(x) = \frac{\sin x}{x}
%    \end{displaymath}, vilket lter formeln visas centrerat p en egen rad
% Om du vill att en formel ska numreras byter du ut displaymath mot equation.
% Det finns massor med matematiska symboler, vilket gr att man behver
% ngon liten manual att titta i om man ska konstruera avancerade formler.
% Se slutet p filen fr lite rd om var du kan hitta sdana.
%
%
% INFOGA FIGURER
%
% Fr att infoga en figur kan man gra p fljande stt:
%    \begin{figure}[htb]
%        \includegraphics[scale=0.5, angle=90]{exec_flow.eps}
%        \caption{Detta r bildtexten}
%        \label{EXECFLOW}
%    \end{figure}
% Om man vill referera till denna bild i texten skulle man d skriva enligt:
%    ...i figur \ref{EXECFLOW} kan man se att...
% Ngra sm frklaringar till figurer:
%    [htb] = talar om hur latex ska frska placera bilden (Here, Top, Bottom)
%            Om du anvnder [!h], innebr det Here!!!
%    scale = kan skala om bilden, om den r skalbar
%    angle = kan rotera bilden
%    exec_flow.eps = filnamnet p bilden. Notera att formatet .EPS anvnds
% Fr att skapa figurer anvnds lmpligtvis programmet xfig:
%    bergner@peppar:~/edu/sysprog/lab1> xfig &
% Rita (och spara ofta) tills du r klar. Vlj sedan "Export" och exportera
% din figur till EPS-format.
% Om man vill kan man anvnda endast \includegraphics, men det r inte ofta
% man gr det.
%
%
% INFOGA TABELLER
%
% Om man vill skapa en tabell gr man p fljande stt:
%    \begin{table}[htb]
%        \begin{tabular}{|rlp{10cm}|}
%            \hline
%            13 & $17.26$ & En kommentar som kan strcka sig ver flera rader \\
%            \hline
%        \end{tabular}
%        \caption{Tabelltexten...}
%        \label{TBL:MINTABELL}
%    \end{table}
% Om man vill kan man endast anvnda raderna 2-6, dvs f en tabell utan text
% och nummer. Om man gr p detta vis kommer tabellen alltid att lggas p
% det stlle den skrivs i koden, dvs ungefr samma sak som [!h] -> Here!!!
% Ngra frklaringar:
%    l, r, c = vnsterjustera, hgerjustera eller centrera kolumn
%    p{bredd} = skapa en vnsterjusterad kolumn med en viss bredd
%               kan innehlla flera rader text
%    | = en vertikal linje i tabellen
%    \hline = en horisontell linje i tabellen
%    & = kolumnseparator
%    \\ = radseparator
% Tnk p att tabeller oftast ser bttre ut med ganska f linjer.
%
%
% INFOGA KLLKOD ELLER UTDATA FRN TESTKRNINGAR
%
% Om man vill infoga kllkod eller ngot annat liknande, t.ex. utdata frn
% en testkrning r det bra om LaTeX terger utdatan korrekt, dvs en radbrytning
% betyder en radbrytning och 8 mellanslag p rad betyder 8 mellanslag p rad.
% Fr att stadkomma detta anvnds:
%    \begin{verbatim}
%        allt som skrivs hr terges exakt, med skrivmaskinstypsnitt
%    \end{verbatim}
% Oftast finns det dock bttre verktyg fr att skriva ut kllkod. Exempel p
% sdana r a2ps, enscript och atp.
%
%
% NDRA STORLEK P TEXT
%
% Om du vill ndra storleken p ett stycke, t.ex. p din nyss infogade
% testkrning omger du stycket med \begin{STORLEK} \end{STORLEK}, dr
% STORLEK r ngon av:
%    tiny, scriptsize, footnotesize, small, normalsize, large, Large,
%    LARGE, huge, Huge
% Tnk p att inte mixtra fr mycket med storlekar bara.
%
%
% SKAPA LISTOR AV OLIKA SLAG
%
% Det r ganska vanligt att man vill rada upp saker p ngot stt. Fr att
% skapa punktlistor anvnds:
%    \begin{itemize}
%        \item Detta r frsta punkten
%        \item Detta r andra punkten
%    \end{itemize}
% Om man istllet vill ha en numrerad lista kan man anvnda enumerate istllet
% fr itemize. Listor kan anvndas i flera niver
%
%
% MER INFORMATION OM LaTeX
%
% Lite blandad information om LaTeX, lnkar och annat hittar du p
% http://www.cs.umu.se/~bergner/latex.htm
% En del information om rapportskrivning hittar du p
% http://www.cs.umu.se/~bergner/rapport/
% Det finns massor med information om LaTeX p Internet. Ett litet urval:
% http://www.giss.nasa.gov/latex/
%     r en mycket vlfylld sida om LaTeX
% http://wwwinfo.cern.ch/asdoc/WWW/essential/essential.html
%     r en manual som genererats utifrn ett LaTeX-dokument mha latex2html
% http://tex.loria.fr/english/
%     r ett fylligt arkiv av lnkar till LaTeX-dokument p Internet
%
% Min personliga favorit r dock manualen "The Not So Short Introduction to
% LaTeX2e", som finns i DVI-format p ~bergner/LaTeX/lshort2e.dvi
% Dr str i princip allt man behver veta. Det r bara att anvnda xdvi och
% titta efter det du sker, vilket oftast finns dr.
% Om du, precis som jag, vill kunna leka med mnga kommandon i LaTeX finns en
% "LaTeX Command Summary" p ~bergner/LaTeX/latexcmds.ps

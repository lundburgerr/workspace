%1. How to compile the program
To compile the disassembler, run \texttt{make dismips}.
%2. How to run the program
Run the disassembler by executing the command \texttt{dismips file}.
Where \texttt{file} is the path to the file containing the MIPS32 machine code to be disassembled.
The disassembler supports all instructions except those with op-code 16, 17 and 18 except for the commands
\texttt{bclf}, \texttt{bclt}, \texttt{mfc1} and \texttt{mtc1} which are included.

%3. How is the output formatted
The disassembler outputs to \emph{standard out} the following for each line of machine code:
\begin{enumerate}
\item the original instruction.
\item the type of instruction.
\item the values for relevant fields for that type of instruction in decimal.
\item the values for relevant fields for that type of instruction in hexadecimal.
\item and the mnemonic representation.
\end{enumerate}
in the format \texttt{1; 2; 3; 4; 5;} where the numbers corresponds to the information in the list above.
The disassembler will output error messages that corresponds to whatever type of error is in the machine code e.g. if you try to use a floating point instruction it will tell you it is not allowed or if you use an invalid op- or function-code it will also tell you that---and then continue with the next line of machine code.
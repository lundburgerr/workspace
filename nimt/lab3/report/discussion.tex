The results of sections \ref{sec:novel} and \ref{sec:harmonic} show that the new model derived in the theory fits much better to the measured values than the simple model of the damped harmonic oscillator whose amplitude decays exponentially. Comparing fig. \ref{fig:resultExp} and \ref{fig:resultTan} it is visible that the deviations are less particularly for large values of $t$. However, it was not possible to get quantitative results from the theory neither for the pendulum \emph{with more air drag} nor for the \emph{heavy} pendulum. 

An increase of mass should theoretically increase the friction if all other parameters are unchanged\footnote{To be sure that $R$ is unchanged we fixed the screws around the center of mass of the pendulum. }. But this could not be observed. There was even a tendency that the damping decreased with an increase of mass. The reason for this could be that the pendulum motion is not entirely constrained to be around the joint axis (z-axis in fig. \ref{fig:forces}) but it wobbles along this direction as well. This may cause an energy loss that is much higher that the ones describes in the theory. Adding mass straightens the plane of oscillation so that the wobbling becomes fewer and thus the energy loss decreases. 

As shown in section \ref{sec:airdrag} the results for the pendulum \emph{with more air drag} are qualitatively good but again we cannot make a quantitative statement because the measured damping does not change as expected with the change of the cross sectional area. In the case the deviation might be caused by the theoretical assumptions. For the air drag eq. (\ref{eq:coulomb}) is only applicable for a rigid body moving through a fluid. The idea to apply it on infinitesimal area elements $dA$ was therefore comparable to dividing the cardboard in many rigid bodies. This might not have been a good assumption if one takes into consideration that for the total air flow always the 3-dimensional shape of the whole body matters. The fluid dynamics of the pendulum might be much more complex than the theory takes into consideration. 

The calibration that was made in section \ref{sec:calibration} did not give a satisfying result. The uncertainty is even higher than the value itself. This is obvious regarding fig. \ref{fig:calibration} since a small change in $A$ produces a large change in $d$ comparable with the behaviour of a scissor. Another calibration method would be to move the pendulum to $\varphi=\pi / 2$ and then calculate its initial energy as $E_{init}=RMg$. The first measured energy could then be associated with the initial energy since the loss for a quarter period is neglectable. Due to time constraints we did not apply this method.

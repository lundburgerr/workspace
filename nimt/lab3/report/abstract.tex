%1. What did we do??
In this lab we measured the decay of energy in a pendulum, both from friction and air drag.
%2. experimental design
We used a laser \emph{time of flight} (ToF) method to determine the velocity--and thereby the energy---of
a pendulum; in order to examine the decay of energy against time.
%3. The results were
	%3.1 Decay
We fitted the measured decay in energy to an exponential decay model and a novel model. 
	%3.2 Air drag
The measurements with increased air drag qualitatively confirmed the theoretical expectation, however no quantitative statement could be made. 
	%3.2 Calibration
	
	%3.3 Weird damping with weight
The measured data did not fit the prediction that a heavier pendulum is damped faster, so there was an unknown factor that we could not take into account.
%4. What can we do better, conclusions.
The unknown factor of the pendulum we believe was caused by the pendulum not being completely 
constrained to move around one axis. Without this issue, the loss of energy due to friction could be better characterized.
We also didn't fit our model to a more complicated model that we derived.

%1. The problem in general: determine movement, what methods is there. Mention non-invasive
One way to determine the small movements of objects, is to attach an accelerometer or similar 
electronics on the object. 
But the problem with this method is that the sensor itself will affect the object---it is an invasive method. Ideally we want to use a measuring technique that is \emph{non-invasive}.
One way to accomplish this is to use the laser triangulation technique.

%2. Describe how laser triangulation works
The laser triangulation technique works by pointing a laser at the moving object and then collect the scattered light onto some detector. From this the position of the object can be determined at different times; the choice of detector depends on the application. 
%3. How good is laser triangulation today? accuracy etc.
But in general this technique can today offer resolution in distance of $\sim nm$ %TODO insert reference
and measure frequencies of up to $\sim 100kHz$. %TODO insert reference

%4. Other applications
The laser triangulation technique has a wide area of application
including a 3D-scanning device \cite{Franca05},
biofilm thickness measurements \cite{Okkerse00} and fast optical hazard detection for planetary rovers \cite{Matthies97}.
%5. Limitations
The limitations of this technique mostly depends on what detectors you use but in general the system
can be expensive with sensors and laser costing several \$100 or \$1000. The system might also be rather large and not suited for measurements outside the laboratory.

%6. Our specific problem
In this lab we applied this technique to measure a tuning fork to determine the following properties: main frequency, harmonic frequencies and amplitude.
As a detector we used a \emph{position sensitive detector} (PSD). The PSD outputs
the position of the incident light on the PSD.
The advantages of this sensor is that it can measure very high frequencies and very fast. The resolution of the sensor is $\sim 10\mu m$ and it's also very easy to use in electronics.%TODO Check the values for the resolution in the datasheet
The main disadvantage with this sensor is that the position is determined by some moving average of the intensity of light hitting it; so if the beam is not symmetric in two dimensions, but warped in some way; this can give erroneous results for the position.
%1. What did we do??
In this lab we used an inductive method (using a coil) to detect cracks in metals and measure the resistivity of metals.
%2. experimental design
Since there is a power loss in the metals when we induce currents to it. There should be an increased resistance in the coil which we can measure; from this the resistivity can be determined.
To detect cracks we assume that the current can't flow through a crack. That means lower induced current and less power loss seen in the coil.
%3. results and conclusions
At currents through the
coil with frequency of $f=1\;kHz$ both resistance and reactance increase in the coil near ferromagnetic materials which agrees with our theory. However at $f=100\;kHz$ the reactance decreases.
There was a noticeable change in output close to cracks. And there was also a noticeable difference in
output when the coil were close to different conductors indicating that the method is well suited for measuring resistivity. However we did not find a mathematical model that fit well with the measurement.

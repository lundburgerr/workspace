\subsection{Crack Detection}
At the lower frequency of $f=1$ kHz, our results (see fig.\ref{fig:crack1kHz}) agree with the theoretical expectations for ferromagnetic materials (see fig.\ref{fig:CoilImpedance}): both resistance and reactance of the coil rise in the presence of the material. At a crack, the reactance continues to rise, while the resistance decreases. %Based on other properties (look, weight) of the investigated material, the sample is probably comprised of steel.
\par
However, at higher frequencies the resistance at a crack rises as well, suggesting a power dissipation due to the crack.
\par
At a frequency $f=100$ kHz, the reactance decreases when the probe coil is situated on the material. This indicates that the material is then non-ferromagnetic. A possible reason for this behaviour might lie in the small penetration depth of the magnetic field into the material at a high frequency. Here, the magnetic permeability $\mu$ of the material is reduced to almost unity\cite{feromagnetism}, inhibiting the increase in strength of the primary magnetic field. 

\subsection{Resistivity Measurement}
We think the results were poor because the theory did not fit well with reality.
We don't think the deviation to the constant in table \ref{table:measurements} is due to any measurement error since the set up used could measure in nano-volts; also the values in table \ref{table:measurements} for the constant C was higher for lower resistivity, so it followed a pattern which it should not. 
We are not sure about the dependence of $\rho$ on the \emph{assumed constant} $J_s$ in eq. (\ref{eq:th_Js}); intuitively we assumed that it might be proportional to $1/\rho$, but that will still fit reality even worse.
However there is differences in the measured values for the voltages for different materials with different resistivity
so this indicates this method can be used to measure resistivity.
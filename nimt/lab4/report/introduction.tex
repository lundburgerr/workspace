%1. The problem in general: determine coefficient of thermal expansion of metal
%TODO

%2. Describe a bit about interferometry (Michelson interferometer) and heat expansion
%2.1 michelson interferometer
The Michelson interferometer is a common configuration for optical interferometry. A beam of light is split
into to two and made to travel different paths. The first path is used as reference and held fixed, whereas the second path is made to vary. When the beams from both these paths are joined together they will interfere with each other. The interference pattern can be measured over time and the difference in path length can be determined.
%2.2 michelson interferometer, how good is it; accuracy? 
If the path difference between the two paths changes by half the wavelength of light, we will move between two maxima in the interference pattern. So we can detect differences in path length that is less than the wavelength of light. We can also measure larger path differences by counting how many peaks you have seen.
%TODO above seems a bit not perfect

%2.2 heat expansion
%TODO

%3. How good is something today --- Dont do this

%4. Other applications --- dont do this
%TODO

%6. Our specific problem
In our lab we want to measure the coefficient of thermal expansion of unknown metal rods and from this identify what metal it is. We predict that the coefficient can be measured accurate enough to determine
what metal we are testing on. We believe that the expansion of the metal rod can be measured very exactly.
However it will not be as easy to measure the temperature of the rod since we can only measure a few points on the outside of the rod and we believe this will be the most significant source of error.
%Function: The function of the Discussion is to interpret your results in light of what was already known about the subject of the investigation, and to explain our new understanding of the problem after taking your results into consideration. The Discussion will always connect to the Introduction by way of the question(s) or hypotheses you posed and the literature you cited, but it does not simply repeat or rearrange the Introduction. Instead, it tells how your study has moved us forward from the place you left us at the end of the Introduction.

%1. Do your results provide answers to your testable hypotheses? If so, how do you interpret your findings?

%2. Do your findings agree with what others have shown? If not, do they suggest an alternative explanation or perhaps a unforseen design flaw in your experiment (or theirs?)

%3.Given your conclusions, what is our new understanding of the problem you investigated and outlined in the Introduction?


%4. If warranted, what would be the next step in your study, e.g., what experiments would you do next?
To get a better picture of the expected results of this experiment we identified the metals the rods were constituted of. This was done by measuring the densities of both rods and comparing it to tabulated values. From this we found rod 1 to be made out of aluminium and rod 2 to be made out of titanium.\\

After the rods where identified we could also look up their respective linear thermal expansion coefficients.
For the aluminium rod the tabulated value were $23\cdot10^{-6} \rm{1/K}$ \cite{coff}. This is outside the measured value of $26.605 \cdot 10^{-6} \pm 0.416 \cdot 10^{-6} \; \rm{1/K}$ which indicate some systematic error.
For the titanium rod which has a tabulated coefficient of linear thermal expansion of $8.6\cdot10^{-6} \rm{1/K}$ \cite{coff}; we see that our calculated value of about $19.371 \cdot 10^{-6} \pm 0.110 \cdot 10^{-6} \; \rm{1/K}$ is not reasonable either.\\

One of the main reasons why our measurements suffered significant errors could possibly be due to an uneven temperature gradient in both the radial and axial direction. As a result the temperature measurements we got from the thermistors will not give a good representation of the rods true temperature.

Main programmet kommer sköta initialisering av variabler, mutex och tolkningen av inargument.
Sedan så skapar huvudtråden N-1 trådar som kör funktionen \texttt{taskManager} som i sin tur kallar på \texttt{mfind\_p}. Huvudtråden kommer sedan själv köra taskManager och när alla trådar är färdiga så städar huvudtråden upp efter processen.
Det behövs fyra globala variabler för att sköta synkningen mellan trådarna och dom är:
\texttt{waiting} som anger hur många trådar som inte kör \texttt{mfind\_p}, utan väntar på att en task ska läggas till i taskpoolen; \texttt{blocked} som är en array där varje element anger huruvida den motsvarande tråd väntar på en task eller inte, detta är för att förhindra att en tråd ökar på \texttt{waiting} flera gånger då tråden inte kommer sluta loopa bara för att den inte har någon task; \texttt{quit} som signalerar till trådarna att nu kan dom avsluta funktionen för sökningen är färdig, detta kommer ske när \texttt{waiting} är lika med antalet trådar; och till sist en \texttt{taskpool} som innehåller relativa sökvägar till dom mappar som behövs gås igenom.

%TODO: Beskriv dom mutexar som används
\def\accessTaskPoolMutex{accessTaskPool}
\def\getTaskMutex{getTask}
\def\initThreadIDMutex{initThreadID}
Programmet använder tre mutexar för att hantera synkroniseringen mellan trådar.
\texttt{\initThreadIDMutex} används för att ge varje tråd ett unikt id mellan 0 och $N-1$ om $N$ är antalet trådar.
\texttt{\accessTaskPoolMutex} används för att förhindra att olika trådar försöker komma åt \texttt{taskpoolen} samtidigt vilket kan ge odefinierat beteende.
\texttt{\getTaskMutex} används för att kunna bestämma när trådarna kan avsluta.

\Subsection{taskManager}
\begin{enumerate}
\item \texttt{count = 0}
\item	Ge tråden ett unikt \texttt{id}.
\item	Kör loop tills färdig.
	\begin{enumerate}
	\item	Lås \getTaskMutex.
		\begin{enumerate}
		\item	Lås \accessTaskPoolMutex.
			\begin{enumerate}
			\item	\texttt{mapp = getFolderFromTaskPool()}
			\end{enumerate}
		\item	Lås upp \accessTaskPoolMutex.
		\item 	Om \texttt{mapp == NULL} och om \texttt{!blocked[id]}. Öka då på \texttt{waiting} med ett och sätt \texttt{blocked[id] == true}.
		\item	Om \texttt{mapp != NULL} så sätt \texttt{waiting = 0} till 0 och alla element i \texttt{blocked} till \texttt{false}.
		\item 	Om \texttt{waiting == N} så sätt \texttt{quit} till \texttt{true}
		\end{enumerate}
	\item	Lås upp \getTaskMutex.
	\item	Om \texttt{mapp == NULL} och \texttt{quit == true}
		\begin{enumerate}
		\item	Gå ut ur loopen.
		\end{enumerate}
	\item	Om \texttt{mapp == NULL} och \texttt{quit == false}
		\begin{enumerate}
		\item	Börja om loopen.
		\end{enumerate}
	\item	Om \texttt{mapp != NULL}
		\begin{enumerate}
		\item	Kör \texttt{mfind\_p} med \texttt{mapp} som inargument.
		\item	Öka på \texttt{count} med ett.
		\end{enumerate}	
	\end{enumerate}
\item	Skriv ut tråd-id och \texttt{count}.
\item	Avsluta funktionen.
\end{enumerate}


\Subsection{mfind\_p}
Funktionen \texttt{mfind\_p} har som argument en sökväg till en mapp och två variabler som används för att
bestämma om en fil är den man söker efter.
\begin{enumerate}
\item	Öppna \texttt{mapp} på given sökväg.
\item 	Gå igenom alla filer i mappen.
	\begin{enumerate}
	\item	Kolla om filen är den du söker efter. Skriv i så fall ut dess sökväg på standard output.
	\item 	Om filen är en mapp och inte "." eller "..", gör följande
		\begin{enumerate}
		\item	Lås \accessTaskPoolMutex
			\begin{enumerate}
			\item	Lägg till ny mapp i \texttt{taskpool}
			\end{enumerate}
		\item	Lås upp \accessTaskPoolMutex	
		\end{enumerate}
	\end{enumerate}
\item	Stäng mappen.
\item 	Avsluta funktionen.
\end{enumerate}

%______________________________________________________
%
%   LaTeX-mall fr nybrjare
%
%   Konstruerad av Marcus Bergner, bergner@cs.umu.se
%
%   Vid funderingar titta lngst ned i denna fil,
%   eller skicka ett mail
%______________________________________________________
%

% lite instllningar
\documentclass[10pt, titlepage, oneside, a4paper]{article}
\usepackage{amsmath}
\usepackage[T1]{fontenc}
\usepackage[utf8]{inputenc}
\usepackage[english]{babel}
\usepackage{amssymb, graphicx, fancyheadings}
\usepackage{placeins}

%For subfigures
\usepackage{caption}
\usepackage{subcaption}
%For including matlab code
%\usepackage{mcode}

\addtolength{\textheight}{20mm}
\addtolength{\voffset}{-5mm}
\renewcommand{\sectionmark}[1]{\markleft{#1}}

% \Section ger mindre spillutrymme, anvnd dem om du vill
\newcommand{\Section}[1]{\section{#1}\vspace{-8pt}}
\newcommand{\Subsection}[1]{\vspace{-4pt}\subsection{#1}\vspace{-8pt}}
\newcommand{\Subsubsection}[1]{\vspace{-4pt}\subsub\section{#1}\vspace{-8pt}}

% appendices, \appitem och \appsubitem r fr bilagor
\newcounter{appendixpage}

\newenvironment{appendices}{
	\setcounter{appendixpage}{\arabic{page}}
	\stepcounter{appendixpage}
}{
}

\newcommand{\appitem}[2]{
	\stepcounter{section}
	\addtocontents{toc}{\protect\contentsline{section}{\numberline{\Alph{section}}#1}{\arabic{appendixpage}}}
	\addtocounter{appendixpage}{#2}
}

\newcommand{\appsubitem}[2]{
	\stepcounter{subsection}
	\addtocontents{toc}{\protect\contentsline{subsection}{\numberline{\Alph{section}.\arabic{subsection}}#1}{\arabic{appendixpage}}}
	\addtocounter{appendixpage}{#2}
}

% ndra de rader som behver ndras
%\def\inst{Department of computer science}
\def\inst{Department of Mathematics and Mathematical Statistics}
\def\typeofdoc{Lab report}
\def\course{Numerical methods for PDE 7.5hp}
\def\pretitle{Laboratory 3}
\def\title{Piecewise Linear Approximations and the Finite Element Method}
\def\name{Robin Lundberg}
\def\username{rolu0008}
\def\email{\username{}@student.umu.se}
\def\path{}
\def\graders{Patrik Norqvist, Christian Engström and Juan Carlos Araujo-Cabarcas.}


% om du vill referera till katalogen dr dina filer ligger kan du 
% anvnda \fullpath som kommer att vara "~username/edu..." o.s.v.
%\def\fullpath{\raisebox{1pt}{$\scriptstyle \sim$}\username/\path}


% Hr brjar sjlva dokumentet
\begin{document}

	% skapar framsidan (om den inte duger: gr helt enkelt en egen)
	\begin{titlepage}
		\thispagestyle{empty}
		\begin{large}
			\begin{tabular}{@{}p{\textwidth}@{}}
				\textbf{UMEÅ UNIVERSITY \hfill \today} \\
				\textbf{\inst} \\
				\textbf{\typeofdoc} \\
			\end{tabular}
		\end{large}
		\vspace{10mm}
		\begin{center}
			\LARGE{\pretitle} \\
			\huge{\textbf{\course}}\\
			\vspace{10mm}
			\LARGE{\title} \\
			\vspace{15mm}
			\begin{large}
				\begin{tabular}{ll}
					\textbf{Name} & \name \\
					\textbf{E-mail} & \texttt{\email} \\
					%\textbf{Path} & \texttt{/home/rolund03/workspace/gsl\_ode} \\
				\end{tabular}
			\end{large}
			\vfill
			\large{\textbf{Supervisor}}\\
			\mbox{\large{\graders}}
		\end{center}
	\end{titlepage}


	% fixar sidfot
	\lfoot{\footnotesize{\name, \email}}
	\rfoot{\footnotesize{\today}}
	\lhead{\sc\footnotesize\title}
	\rhead{\nouppercase{\sc\footnotesize\leftmark}}
	\pagestyle{fancy}
	\renewcommand{\headrulewidth}{0.2pt}
	\renewcommand{\footrulewidth}{0.2pt}

	% skapar innehllsfrteckning.
	% Tnk p att kra latex 2ggr fr att uppdatera allt
	\pagenumbering{roman}
	%\tableofcontents
	
	% och lägger in en sidbrytning
	\newpage

	\pagenumbering{arabic}

	% i Sverige har vi normalt inget indrag vid nytt stycke
	\setlength{\parindent}{0pt}
	% men dremot lite mellanrum
	\setlength{\parskip}{10pt}

	% lägger in rubrik (finns \section, men då får man mycket spillutrymme)
	\FloatBarrier
	\Section{Theory}
	\subsection{Time of Flight Methods}

In a time of flight measurement the time $\Delta t=t_2-t_1$ is measured that an object needs to travel from one position $x_1$ to another position $x_2$. If the distance between the two positions $\Delta x=x_2-x_1$ is known, the mean velocity between the two positions is

\begin{equation}
v_{mean}=\frac{\Delta x}{\Delta t}.
\end{equation}

\subsection{The Compound Pendulum}

A compound pendulum consists of a rigid body with a moment of inertia $I$ that is swinging around a pivot. Its movement can be completely described by considering the angle $\varphi$ between the vector from the pivot to the center of mass $\vec{R}$ and the gravitational force $\vec{F}_g$ (in case a visualisation is needed see fig. \ref{fig:forces} in the following section). The force $\vec{F}_g$ exerts a back driving torque $\tau = (\vec{R} \times \vec{F_g})_z = -R F_g \sin \varphi$ on the body. For small angles $\varphi$ one can approximate $sin\varphi \approx \varphi$. Hence the angular equation of motion is

\begin{equation}
I\ddot{\varphi} = -R F_g \varphi.
\end{equation}
 
This is the equation for a simple harmonic oscillator with the angular frequency $\omega = \sqrt{\frac{R F_g}{I}}$ considering $F_g=Mg$ where $M$ is the mass of the body and $g$ is the gravitational constant. The solution to this equation is

\begin{equation}\label{eq:sho}
\varphi(t) = \varphi _0 \cos\left(\omega t + \delta\right).
\end{equation}

The assumption $\delta=0 \Rightarrow \varphi(t=0)=\varphi_0$ is made in the following discussion since the phase is not important in this context.

Now one can ask which energy is stored in such an oscillation. Assuming that at $\varphi=0$ the energy is only kinetic energy all the energy of the system at turning point $\varphi=\varphi_0$  should only consist of potential energy in the gravitational field. This potential energy is equal to the negative of the work that has to be performed \cite{torque} against the back driving torque $\tau$ if one moves the pendulum from the position  $\varphi=0$ to $\varphi=\varphi_0$ on get

\begin{equation}
W = \int_0^{\varphi_0} \tau(\varphi) d\varphi = -\int_0^{\varphi_0} R F_g \varphi d\varphi = -\frac{1}{2}
R F_g \varphi_0^2.
\end{equation}

So the relation between the energy of the oscillation $E=-W$ and its amplitude $\varphi_0$ is

\begin{equation}\label{eq:ephi}
\varphi_0 = \sqrt{\frac{2E}{R F_g}}.
\end{equation} 

This can now be plugged into eq. (\ref{eq:sho}). Due to the damping described in the following sections the amplitude will decay over time so that there is a time dependence $\varphi_0(t)$ and thus also the energy $E(t)$ will flow out of the system with time $t$. The assumption that the angular frequency $\omega$ stays constant during this process is empirically applicable for weakly damped oscillations; which gives that

\begin{equation}\label{eqphi}
\varphi(t) = \sqrt{\frac{2 E(t)}{R F_g}}  \cos\left(\omega t \right).
\end{equation}

If one differentiates this term it can be assumed that $E(t)$ varies much slower with time than $\cos\left(\omega t \right)$ which corresponds to the observation that $\varphi$ changes rapidly from $-\varphi_0$ to $+\varphi_0$ due to the oscillation with  $\omega$ whereas the decrease  $\varphi(t)>\varphi(t+T)$ is much smaller. So one can write that

\begin{equation}\label{eqphi_dot}
\dot{\varphi}(t) \approx  -\omega \sqrt{\frac{2 E(t)}{R F_g}} \sin\left(\omega t \right).
\end{equation}


\begin{figure}[h]
\begin{center}
\includegraphics[scale=0.55]{img/forces.png}
\end{center}\caption{Geometry of the pendulum and forces acting during the oscillation with corresponding cylindrical coordinate system. (Vectors are bold)}\label{fig:forces}
\end{figure}


\subsection{The damping torque $\tau_f$ due to friction at the suspension point}

At the surface of the rod where the pendulum is suspended a friction force $F_f$ occurs during the oscillation. In Coulomb's model of friction \cite{friction} the friction force is considered as being proportional to the normal force $F_N$ that keeps the pendulum on its track. Since the pendulum is continuously moving, the proportionality constant is given by the coefficient of kinetic friction $\mu_k$ that depends on the used materials\footnote{Actually the pendulum stops in the turning points for a very short time. But as long as the coefficient of static friction $\mu_s$ is not much larger than $\mu_k$ this should not have a visible effect. For most materials $\mu_s$ and $\mu_k$ are in the same order of magnitude.}. So one gets the equation

\begin{equation}
F_f=\mu_k F_N.
\end{equation}

Regarding fig. \ref{fig:forces} it becomes clear that the normal force consists of two parts $F_N=F_{N1}+F_{N2}$ where:

\begin{itemize}
\item The force $F_{N1} = F_g \cos \varphi $  counters the component of the gravitational force that is acting in the direction of $\vec{R}$.
\item The force $F_{N2} = M \dot{\varphi}^2 R$  acts as a centripetal force to keep the pendulum on its circular path. \cite{centripetalforce}
\end{itemize}

If $a$ is the radius of the rod where the pendulum is fixed (see again figure \ref{fig:forces}), then the resulting torque due to the friction force is

\begin{equation}\label{eq:tauf}
\tau_f = a F_f = a \mu_k M \left(g \cos \varphi + \dot{\varphi}^2 R\right) \approx a \mu_k M \left(g - \frac{g}{2} \varphi^2  + \dot{\varphi}^2 R\right).
\end{equation}





\subsection{The damping torque $\tau_d$ due to the air drag}

Since the pendulum is moving quite fast and its shape is approximately circular so the air flowing around it should give rise to a lot of curls. In this case one can apply the drag equation \cite{airdrag} 
\begin{equation}\label{eq:coulomb}
F_d=\frac{1}{2} \rho v^2 C_D A.
\end{equation}

In this equation the following parameters occur:

\begin{itemize}
\item $\rho$ is the density of the fluid (here air).
\item $v$ is the velocity of the floating object.
\item $A$ is the cross-sectional area of the object.
\item $c_d$ is the \emph{air drag coefficient} that depends on the shape of the object.
\end{itemize}

However, this equation cannot be applied directly to the pendulum since each infinitesimal cross-sectional area element $dA=drdz$, see fig. \ref{fig:forces}, of it moves with a different velocity $v = \dot{\varphi}r$ depending on the distance $r$ from the pivot. So the infinitesimal force acting on $dA$ is given by $dF_d = \frac{1}{2} \rho  C_d  \dot{\phi}^2   r^2 dr dz$. It gives rise to an infinitesimal torque $d\tau = r dF_d$. 


If $L$ is the length of the pendulum ($\neq R$) and $d$ is its width the total torque due to the drag force is

\begin{equation}\label{eq:taug}
\tau_d = \int d\tau_d =  \frac{1}{2} \rho \dot{\phi}^2 C_d \int_0^L r^3 dr \int_0^d dz = \frac{1}{8}  \rho \dot{\varphi}^2  C_d L^4 d.
\end{equation}

\subsection{Determining the Time dependence of Energy $E(t)$}

Due to the frictional torques $\tau_D$ and $\tau_f$ derived in the sections above (eq. (\ref{eq:tauf}) and (\ref{eq:taug})) the energy of the oscillating System decreases. The ratio of this decrease $dE/dt$ is equal to the power $P=\left( \tau_D + \tau_f \right) \dot{\varphi}$ that the torques take out of the system. 
At this point it is crucial to consider that the equations \ref{eq:tauf} and \ref{eq:taug} only contain information about the absolute value of the torques. Since these torques describe friction, they should always act against the direction of $\dot{\varphi}$. So one has to add this information to get the correct result

\begin{equation}
P=-\rm{sign}\left(\dot{\varphi}\right) \left( \tau_D + \tau_f \right) \dot{\varphi} = -\left( \tau_D + \tau_f \right) \left|\dot{\varphi}\right|.
\end{equation} 

Now one can plug in $\tau_D$ and  $\tau_f$ respectively. One gets the differential equation

\begin{equation}
\frac{dE}{dt} = - \left( \frac{1}{8}  \rho \dot{\varphi}^2  C_D L^4 d + a \mu_k M \left(g - \frac{g}{2} \varphi^2  + \dot{\varphi}^2 R\right) \right)\left|\dot{\varphi}\right|.
\end{equation}

Defining the constants 
\begin{align}
A_1 &= \omega^3 \left(\frac{2}{R F_g}\right)^{3/2} \left(\frac{1}{8}  \rho  C_D L^4d+a \mu_k M R\right) \nonumber \\
A_2 &= \omega \left(\frac{2}{R F_g}\right)^{3/2}a \mu_k M \frac{g}{2} \nonumber \\
A_3 &= \omega \left(\frac{2}{R F_g}\right)^{1/2} a \mu_k M g \nonumber
\end{align}
 and plugging in eq. (\ref{eqphi_dot}) and (\ref{eqphi}) one gets

\begin{equation} \label{eq:full}
\frac{dE}{dt} = - A_1 E^\frac{3}{2} \left|\cos^3(\omega t)\right| - A_2 E^\frac{3}{2} \left|\cos(\omega t)\right|\sin^2(\omega t) - A_3 E^\frac{1}{2} \left|\cos(\omega t)\right|.
\end{equation}

This equation cannot be solved analytically but the program Mathematica can solve it numerically. In fig. \ref{fig:math} the result is visible. The right graph was calculated with with a higher angular frequency omega than the left one. 


\begin{figure}[h]
\begin{center}
\includegraphics[scale=0.5]{img/results.png}
\end{center}\caption{Equation \ref{eq:full} solved numerically by Mathematica for two different values of $\omega$.}\label{fig:math}
\end{figure}

One can observe that the irregularities seem to vanish with a higher value for $\omega$. This is understandable from the fact that oscillations with high frequency do only produce very small fluctuations in the solution that are not visible. To get the solution without fluctuations one could take the mean value over time $<\left|\cos(\omega t)\right|> = <\left|\sin(\omega t)\right|> = \frac{1}{2}$. Defining new constants $B_1=\frac{1}{8}\left(A_1+A_2\right)$ and $B_2=\frac{1}{2}A_3$; then eq. (\ref{eq:full}) becomes

\begin{equation} \label{eq:fullshort}
\frac{dE}{dt} = - B_1 E^\frac{3}{2}  - B_2 E^\frac{1}{2}
\end{equation}

and have the analytical solution

\begin{equation}
E(t) = \frac{B_2}{B_1}\tan^2\left(\frac{\sqrt{B_1 B_2}}{2}t+C\right).
\end{equation}

This analytical function can be used later to fit the measured values. 

In case that the damping due to air drag is much bigger than the damping due to friction we can assume that $B_1 \gg B_2$. So eq. (\ref{eq:fullshort}) simplifies to

\begin{equation}
\frac{dE}{dt} = - B_1 E^\frac{3}{2}
\end{equation}


and has the solution

\begin{equation}\label{eq:air}
E(t)=\frac{4}{\left(B_1 t + C\right)^2}.
\end{equation}

%
%\begin{table}[htbp]
%	\centering
%		\begin{tabular}{l|c|c}
%		 $f$ [Hz] (detected)	 &  n			& prediction $f_n$  \\
%			\hline
%			75 & 1 & 85.4 \\
%			170 & 2    		& 170.75 \\
%			270 & 3 		& 256.1 \\
%			341  & 4 & (main) \\
%			450 & 5 & 426.9 \\
%			680  & 8 & 683  \\
%			880  & 10 & 853.8\\
%			1010 & 12 & 1024.5
%		\end{tabular}
%	\caption{Detected frequencies in comparison with the prediction $f_n=\frac{n}{4}f$}
%	\label{tab:density}
%\end{table}
	
	\FloatBarrier
	\Section{Results}
	%1. Plots of measured temp and peaks from interference.

%2. Results for thermal expansion coefficient, assume some distribution of heat.
% Prediction of material according to density, coefficient ...

%3. Error estimates for dt dL and L and coefficient
The experiment was conducted on two different metal rods that we denote as \emph{rod 1} and \emph{rod 2}.

\subsection{Results for Rod 1}
To determine the thermal expansion coefficients of rod 1, we analysed the normalized intensities measured by the photo diode. A sample of this data can be seen below in fig. \ref{fig:results:1}.

\begin{figure}[htb]
	\centering
	\includegraphics[width=0.8\textwidth]{img/Peaks_alu.png}
	\caption{A sample of the normalized intensity data measured by the photo diode. Peaks of the data is marked by circles.}
	\label{fig:results:1}
\end{figure}

\FloatBarrier
The peaks of the intensities were counted to calculate the linear expansion of the rod using eq. \eqref{eq7}.  
Temperature measurements were also analysed, raw data from the three temperature sensors can be seen below in fig. \ref{fig:results:2} and a calculated mean temperature in fig. \ref{fig:results:3}.
\begin{figure}[htb]
	\centering
	\includegraphics[width=0.8\textwidth]{img/temp_alu.png}
	\caption{Temperature measurements from the three thermistors mounted on the rod versus time.}
	\label{fig:results:2}
\end{figure}
\begin{figure}[htb]
	\centering
	\includegraphics[width=0.8\textwidth]{img/mtemp_alu.png}
	\caption{Mean temperature of the rod versus time.}
	\label{fig:results:3}
\end{figure}
From this mean temperature the thermal expansion coefficient of Rod 1, $\alpha_1$, was calculated by using eq. \eqref{eq3}. The value obtained was 
$\alpha_1 =26.605 \cdot{10^{-6}} \; \rm{1/K}$.
The error estimate was calculated using \emph{Gauss propagation of uncertainty} and eq. \eqref{eq4} with
$L=0.280 \; \rm{m}$, $\Delta T = 7.1377 \; \rm{K}$ and $\Delta L = 5.3172 \cdot 10^{-6} \; \rm{m}$ with corresponding errors of $1 \; \rm{mm}$, $0.1 \; \rm{K}$ and $3.165 \cdot 10^{-7} \; \rm{m}$ respectively  to be $\sigma_{\alpha_1} = 0.416 \cdot 10^{-6} \; \rm{1/K}$.

A figure of the temperature dependency of the length expansion can be seen below in fig. \ref{fig:results:4}, assuming that this dependence is linear; eq. \eqref{eq4} is equal to eq. \eqref{eq3} and evaluations can be made for large $\Delta T$ and $ \Delta L$.

\begin{figure}[htb]
	\centering
	\includegraphics[width=0.8\textwidth]{img/dl_alu.png}
	\caption{Length expansion of the rod as a function of 		temperature.}
	\label{fig:results:4}
\end{figure}
\FloatBarrier

\subsection{Results for Rod 2}
The calculations described in the above sections where repeated for the second rod. Below we can find the corresponding figures, fig. \ref{fig:results:5},\ref{fig:results:6},\ref{fig:results:7} and \ref{fig:results:8} for these calculations.\\

\begin{figure}[htb]
	\centering
	\includegraphics[width=0.8\textwidth]{img/Peaks_tit.png}
	\caption{A sample of the normalized intensity data measured by the photo diode. Peaks of the data is marked by circles.}
	\label{fig:results:5}
\end{figure}

\begin{figure}[htb]
	\centering
	\includegraphics[width=0.8\textwidth]{img/temp_tit.png}
	\caption{Temperature measurements from the three thermistors mounted on the rod versus time.}
	\label{fig:results:6}
\end{figure}

\begin{figure}[htb]
	\centering
	\includegraphics[width=0.8\textwidth]{img/mtemp_tit.png}
	\caption{Mean temperature of the rod versus time.}
	\label{fig:results:7}
\end{figure}

\begin{figure}[htb]
	\centering
	\includegraphics[width=0.8\textwidth]{img/dl_tit.png}
	\caption{Length expansion of the rod as a function of 		temperature.}
	\label{fig:results:8}
\end{figure}

\FloatBarrier
For this rod the thermal expansion coefficient was found to be $\alpha_2 = 19.371 \cdot 10^{-6}; 1/K$.
This error estimate was also calculated using \emph{Gauss propagation of uncertainty} and eq. \eqref{eq4}. But with
$L=0.245 \; \rm{m}$, $\Delta T = 30.6110 \; \rm{K}$ and $\Delta L =  1.4527 \cdot 10^{-4} \; \rm{m}$ with corresponding errors of $1 \; \rm{mm}$, $0.1 \; \rm{K}$ and $3.165 \cdot 10^{-7} \; \rm{m}$ respectively  to be $\sigma_{\alpha_2} =  0.110 \cdot 10^{-6} \; \rm{1/K}$.




%	% här brjar alla bilagor. Denna måste finnas med även om bara
%	% bilagor anges i \begin{appendices} ... \end{appendices}
%	\appendix
%
%	\Section{Bilaga 1}
%	\ldots{}ligger direkt i dokumentet
%
%	% bilagor, t.ex. källkod. En tom extrasida kommer att skrivas ut för
%	% att få alla sidnummer att stämma
%	\begin{appendices}
%		\appitem{Källkod}{0}
%		\appsubitem{\texttt{mish.c}}{2}
%		\appsubitem{\texttt{mish.h}}{1}
%		\appitem{En bilaga på 3 sidor}{3}
%	\end{appendices}

\end{document}


% Lite information om hur man arbetar med LaTeX
%-----------------------------------------------
%
% LaTeX-koden kan skrivas med en godtycklig editor.
% Fr att "kompilera" dokumentet anvnds kommandot latex:
%    bergner@peppar:~/edu/sysprog/lab1> latex rapportmall.tex
% Resultatet blir ett antal filer, bl.a. en som heter rapportmall.dvi.
% Denna fil kan anvndas fr att titta hur dokumentet egentligen ser
% ut med hjlp av programmet xdvi:
%    bergner@peppar:~/edu/sysprog/lab1> xdvi rapportmall.dvi &
% Du fr d upp ett fnster som visar ditt dokument. Detta fnster
% kommer automatiskt att uppdateras d du ndrar och kompilerar om din
% LaTeX-kod. 
% Nr du anser att din rapport r frdig att skrivas ut anvnder man
% lmpligtvis kommandona dvips och lpr:
%    bergner@peppar:~/edu/sysprog/lab1> dvips -P ma436ps rapportmall.dvi
% Om man vill ha kvar PostScript-filen som dvips genererar kan man gra:
%    bergner@peppar:~/edu/sysprog/lab1> dvips -o rapport.ps rapportmall.dvi
%    bergner@peppar:~/edu/sysprog/lab1> lpr -P ma436ps rapport.ps
% OBS!!! Fr att innehllsfrteckningen och eventuella referenser till
% tabeller och figurer garanterat ska stmma mste man kra latex 2ggr
% p sitt dokument efter att man har ndrat ngot.
%
%
% Lite information om saker man kan tnkas behva i sitt arbete med LaTeX
%-------------------------------------------------------------------------
%
% FORMATTERA TEXT
%
% Fr att formattera text p lite olika stt kan man anvnda fljande LaTeX-
% kommandon:
%    \textbf{denna text kommer att vara i fetstil}
%    \emph{denna text r viktig (kursiv stil)}
%    \texttt{i denna text blir alla tecken lika breda, som med en skrivmaskin}
%    \textsf{denna text visas med ett typsnitt utan serifer}
%
%
% MATEMATISKA FORMLER
%
% Fr att typstta matematiska formler kan man anvnda:
%    $f(x) = x^2 - 3$, vilket lgger in formeln i texten, eller
%    \begin{displaymath}
%        g(x) = \frac{\sin x}{x}
%    \end{displaymath}, vilket lter formeln visas centrerat p en egen rad
% Om du vill att en formel ska numreras byter du ut displaymath mot equation.
% Det finns massor med matematiska symboler, vilket gr att man behver
% ngon liten manual att titta i om man ska konstruera avancerade formler.
% Se slutet p filen fr lite rd om var du kan hitta sdana.
%
%
% INFOGA FIGURER
%
% Fr att infoga en figur kan man gra p fljande stt:
%    \begin{figure}[htb]
%        \includegraphics[scale=0.5, angle=90]{exec_flow.eps}
%        \caption{Detta r bildtexten}
%        \label{EXECFLOW}
%    \end{figure}
% Om man vill referera till denna bild i texten skulle man d skriva enligt:
%    ...i figur \ref{EXECFLOW} kan man se att...
% Ngra sm frklaringar till figurer:
%    [htb] = talar om hur latex ska frska placera bilden (Here, Top, Bottom)
%            Om du anvnder [!h], innebr det Here!!!
%    scale = kan skala om bilden, om den r skalbar
%    angle = kan rotera bilden
%    exec_flow.eps = filnamnet p bilden. Notera att formatet .EPS anvnds
% Fr att skapa figurer anvnds lmpligtvis programmet xfig:
%    bergner@peppar:~/edu/sysprog/lab1> xfig &
% Rita (och spara ofta) tills du r klar. Vlj sedan "Export" och exportera
% din figur till EPS-format.
% Om man vill kan man anvnda endast \includegraphics, men det r inte ofta
% man gr det.
%
%
% INFOGA TABELLER
%
% Om man vill skapa en tabell gr man p fljande stt:
%    \begin{table}[htb]
%        \begin{tabular}{|rlp{10cm}|}
%            \hline
%            13 & $17.26$ & En kommentar som kan strcka sig ver flera rader \\
%            \hline
%        \end{tabular}
%        \caption{Tabelltexten...}
%        \label{TBL:MINTABELL}
%    \end{table}
% Om man vill kan man endast anvnda raderna 2-6, dvs f en tabell utan text
% och nummer. Om man gr p detta vis kommer tabellen alltid att lggas p
% det stlle den skrivs i koden, dvs ungefr samma sak som [!h] -> Here!!!
% Ngra frklaringar:
%    l, r, c = vnsterjustera, hgerjustera eller centrera kolumn
%    p{bredd} = skapa en vnsterjusterad kolumn med en viss bredd
%               kan innehlla flera rader text
%    | = en vertikal linje i tabellen
%    \hline = en horisontell linje i tabellen
%    & = kolumnseparator
%    \\ = radseparator
% Tnk p att tabeller oftast ser bttre ut med ganska f linjer.
%
%
% INFOGA KLLKOD ELLER UTDATA FRN TESTKRNINGAR
%
% Om man vill infoga kllkod eller ngot annat liknande, t.ex. utdata frn
% en testkrning r det bra om LaTeX terger utdatan korrekt, dvs en radbrytning
% betyder en radbrytning och 8 mellanslag p rad betyder 8 mellanslag p rad.
% Fr att stadkomma detta anvnds:
%    \begin{verbatim}
%        allt som skrivs hr terges exakt, med skrivmaskinstypsnitt
%    \end{verbatim}
% Oftast finns det dock bttre verktyg fr att skriva ut kllkod. Exempel p
% sdana r a2ps, enscript och atp.
%
%
% NDRA STORLEK P TEXT
%
% Om du vill ndra storleken p ett stycke, t.ex. p din nyss infogade
% testkrning omger du stycket med \begin{STORLEK} \end{STORLEK}, dr
% STORLEK r ngon av:
%    tiny, scriptsize, footnotesize, small, normalsize, large, Large,
%    LARGE, huge, Huge
% Tnk p att inte mixtra fr mycket med storlekar bara.
%
%
% SKAPA LISTOR AV OLIKA SLAG
%
% Det r ganska vanligt att man vill rada upp saker p ngot stt. Fr att
% skapa punktlistor anvnds:
%    \begin{itemize}
%        \item Detta r frsta punkten
%        \item Detta r andra punkten
%    \end{itemize}
% Om man istllet vill ha en numrerad lista kan man anvnda enumerate istllet
% fr itemize. Listor kan anvndas i flera niver
%
%
% MER INFORMATION OM LaTeX
%
% Lite blandad information om LaTeX, lnkar och annat hittar du p
% http://www.cs.umu.se/~bergner/latex.htm
% En del information om rapportskrivning hittar du p
% http://www.cs.umu.se/~bergner/rapport/
% Det finns massor med information om LaTeX p Internet. Ett litet urval:
% http://www.giss.nasa.gov/latex/
%     r en mycket vlfylld sida om LaTeX
% http://wwwinfo.cern.ch/asdoc/WWW/essential/essential.html
%     r en manual som genererats utifrn ett LaTeX-dokument mha latex2html
% http://tex.loria.fr/english/
%     r ett fylligt arkiv av lnkar till LaTeX-dokument p Internet
%
% Min personliga favorit r dock manualen "The Not So Short Introduction to
% LaTeX2e", som finns i DVI-format p ~bergner/LaTeX/lshort2e.dvi
% Dr str i princip allt man behver veta. Det r bara att anvnda xdvi och
% titta efter det du sker, vilket oftast finns dr.
% Om du, precis som jag, vill kunna leka med mnga kommandon i LaTeX finns en
% "LaTeX Command Summary" p ~bergner/LaTeX/latexcmds.ps
